\documentclass[a4paper]{article}
\title{A keyword based input facility for FORTRAN programs}
\author{Friedemann Schautz}
\begin{document}
\maketitle
\begin{abstract}
An hopefully easy-to-use and easy-to-extend input facility written
in FORTRAN77 is briefly described. It is meant to provide 
an human readable uniform structure to the input of typical
numerical applications. Keyword, namelist-like constructions,
macro-like variables, comments and input nesting are supported.
\end{abstract} 
\section{User's view}
\subsection{General structure}
The typical input consists of individual lines, which either 
contain a keyword, optionally followed by arguments, a macro definition,
a comment, or name-list like input. In cases where a keyword is associated
with a large number of values, additional lines are used for this data,
usually ended by the word 'end' on a separate line. A backslash ($\backslash$) 
at the end of a line causes it to be joined with the following one and 
to be treated as a single line. Keywords and macro names {\em are} case-sensitive. 

\subsection{Nesting}
The keyword {\tt load \em filename} causes the file {\em filename} to be
parsed ('included') before processing the of current input resumes. At most
eight levels of nesting are allowed (can be changed). 

\subsection{Special lines}
One line comments start with a hash sign ($\#$) and are ignored, as are blank
lines. Lines starting with double quotes are simply printed out after
macros have been expanded. Some special commands can be used to control
the behaviour of the interpreter, they all start with a dot followed by 
some capital letters. 

\subsection{Keywords}
Keywords are used to trigger some action of the program and pass argument 
values. (In fact, they are almost equivalent to subroutine calls -- see
Programmer's view below). Arguments are separated by blanks. Both keyword
and arguments can contain macros, which will be expanded first. 
Arguments can have default values, which are used if the argument is
omitted. Omitting arguments is possible from right to left.  
Keyword names can be abbreviated, provided the abbreviation is not ambiguous.

\subsection{Macros}
Macros are defined using the special keyword {\tt def} followed by
the name of the macro and its definition, e.g.
\begin{verbatim}
def n 7
def ci-file ci.dat
def parameters 1 2 3 4.0  
\end{verbatim} 
and referred to by affixing a dollar sign, where multi-letter
names have to be enclosed in brackets. Examples
\begin{verbatim}
print $n
update $(parameters)
read-file $(ci-file).$n
\end{verbatim} 
(In the last example, {\tt  ci.dat.7} would be passed to {\tt read-file})
The command {\tt printmacros} prints a list of all currently known 
macros. 
\subsection{Namelist like input}
It is possible to 'group' macros using the notation
\begin{verbatim}
&groupname name1 value1 name2 value2  ...  nameN valueN
\end{verbatim}
which is equivalent to saying 
\begin{verbatim}
 def groupname:name1 value1
 def groupname:name2 value2
           ...
\end{verbatim}
\subsection{Indexed macro names}
Single letter macros, like {\tt {\$}i} can occur inside multi-letter
macro names, so that with {\tt def ab7 100.0 } and {\tt def i 7} 
the macro {\tt \$(ab\$i) } would expand to {\tt 100.0}. Together with
loops this can be used to iterate over files, parameter sets, and the like.

\subsection{Loops}
Simple iterations are supported (vaguely  FORTRAN 'do-loops') with the 
syntax
\begin{verbatim}
 loop var lower_bound upper_bound 
    ...
 end
\end{verbatim}
The loop variable is accessible as a normal macro, and the loop body is
re-interpreted in each iteration, i.e. macros are expanded again.  

\subsection{Arithmetics on macros}
Macros can be added, subtracted , muiliplied, and divided, provided they
expand to numbers. A 'csh-like' @-notation is used. Examples (the blanks are important!):
\begin{verbatim}
@ i = $i + 1
@ n = $i * 1000
\end{verbatim}

\section{Programmer's view}
\subsection{Names}
Names of routines and common blocks of this input facility usually 
start with {\tt p2} (historically for parser 2).
\subsection{The read-execute loop}
Unlike a 'conventional' FORTRAN program having subroutine calls chained 
together in a more or less fixed way, the keyword-driven approach amounts to
executing a loop which reads a keyword from the input, executes an associated
routine, reads the next keyword and so on. This loop is terminated be either
end-of-file on the input file or a special keyword like {\tt quit}.

A main program using 'p2' should roughly look like this
\begin{verbatim}
     program foo
c      ... own initialisation ...

c    init the parser
     call p2init
c    optional call to echo input to output
     call echo(iuin,iuout)
c    the read-execute loop
     call p2go(iuin,0)
c    ... own cleaning up ..
     end
\end{verbatim}
\subsection{Introducing new keywords}
A keyword together with its arguments -- either those from input or default
values if arguments were omitted -- is mapped to a subroutine call (inside the
automatically generated subroutine p2call). 
To indicate that a subroutine corresponds to a keyword, a
special comment starting with {\tt C\$INPUT} is used, like in the following 
example:
\begin{verbatim}
     subroutine foosub
C$INPUT foo
CKEYDOC This keyword is an example and does nothing
     .....
     end
\end{verbatim}
Here a new keyword ({\tt foo}) would be created and the subroutine 
{\tt foosub} would be called (with no arguments) if {\tt foo} occurred in the
input. The optional doc comment can be used to attach a brief documentation to
be extracted by a utility program ({\tt vardoc}) which generates a
pretty-printed list of available keywords. A subroutine can be mapped to
different keywords, e.g. with different default values for arguments. See the
next example:
\begin{verbatim}
     subroutine fooarg( ibar, astring, anumber, iflag)
C$INPUT fooa i a d i
CKEYDOC calls fooarg with 4 arguments from input, no defaults
C$INPUT foob i a=hello d=100.d0 i=1
CKEYDOC calls fooarg , 3 arguments having defaults
C$INPUT fooc i a=hello d=100.d0 0
CKEYDOC calls fooarg , 2 arguments having defaults, one the fixed value 0
C$INPUT food inp a=hello d=100.d0 0
CKEYDOC calls fooarg like  fooc, but the current input unit 
CKEYDOC (the one the keyword is read from) is passed as the first argument
     integer ibar, iflag
     character astring*(*)
     double precision anumber
       ...
    end
\end{verbatim}
As the example shows, arguments are denoted by a letter showing their type
(i integer, d double, a string) optionally followed by $=$default. If one
argument has a default value, all arguments right to it must have default
values as well (since arguments can only be omitted from right to left).
Fixed values for arguments can appear anywhere, implying that the input
keyword takes less arguments that the associated subroutine.
If there is a mismatch between the number and/or types of arguments between
the definition and the use in the input, the interpreter will signal an error.

\subsection{Preprocessors and files}
A preprocessor ({\tt genp2\_defaults.awk}) is used to create the routines 
{\tt p2init, p2inid } and {\tt p2call} from the special comments. These
routines are typically located in a file called {\tt p2prog.F}.
The mappings between keywords and routines are assembled in a file called
{\tt commands.p2}. The parser code itself resides in two file called 
{\tt p2\_defaults.F} and {\tt p2etc.F}. The script {\tt vardoc} can be used
to extract keywords and documentation comments to produce a (latex or ascii) list.

\subsection{Accessing macros from within the program}
All macros defined with {\tt def } or the name list like input can be accessed
from within the program, they can be modified as well and new macros can be
created. The two routines used for this purpose are {\tt p2setv} to define a
macro (or change the value of an existing one) and {\tt p2var} to retrieve a
value. In both cases the values are character strings. For the common case
where macros (parameters etc) are supposed to contain numerical values, two
routines are available to retrieve these numbers, {\tt  p2gtid} and {\tt
p2gtfd}, for integer and floating point (double precision) values.
Here is an example how to use {\tt  p2gtid},  {\tt  p2gtfd} being analogous.
\begin{verbatim}
      call p2gtid('qmc:walkers',nwalker,100,1)
CVARDOC Number of walkers
\end{verbatim}
The first argument is the name of the macro. In the example above {\tt
walkers}  is a 'member' of
a namelist group ({\tt qmc}), so it could be given in the input by 
\begin{verbatim}
&qmc walkers 500
\end{verbatim}
or
\begin{verbatim}
def qmc:walkers 500
\end{verbatim}
The next argument is the corresponding FORTRAN variable, followed by a default
value and a flag, specifying what should happen if the macro is not defined
when p2gtid is called. Possible values are: 0 use default value if the macro
is not present, -1 raise an fatal error if the macro is not present, 1 use
default value and define the macro with this value. In cases 0 and 1 a
warning will be issued when the default value is used. 
The optional documentation following {\tt CVARDOC} can again be extracted with the 
{\tt vardoc} utility. 
 
\subsection{Routines that read their own input}
Sometimes a keyword needs a lot of data and it should appear in the input
rather than in a separate file. In such a case the current input unit can be
passed as a keyword argument (unsing {\tt inp} instead of the type specifier,
as in the foo-example above) and the routine can continue reading its own
data. To be consistent with the rest it should, however, not use {\tt
read(..)} but a call to {\tt gpnln2}. This routine will read a line from the
input (or some other FORTRAN unit), expand macros and spilt the line into
fields (much like awk).  Self-read input should be terminated by the word
'end' on a single line. The functions {\tt ifroms} and {\tt dfroms} can be
used to convert the output of  {\tt gpnln2} to intergers and double precision floats.
The following code fragment demonstrates this use. 
Assuming  an fictitious input fragment to read in some square matrix
\begin{verbatim}
  read_sqmat 3
   1.0 2.0 3.0 
   4.0 5.0 6.0
   7.0 8.0 9.0
  end
\end{verbatim}  
the corresponding routine could look like
\begin{verbatim}
      subroutine rdsqmt(iu,ndim)
C$INPUT read_sqmat inp i
        ....
      include 'inc/p2_dim.inc'
      include 'inc/p2.inc'
      dimension idx1(MXF),idx2(MXF)
        ....
      do i=1,ndim
        call gpnln2(iu,il,lne,MXLNE,idx1,idx2,nf,lbuf,'rdsqmt')
        if(nf.ne.ndim) then
          call fatal('read_sqmat: wrong number of elements')
        else
          do  j=1,ndim
            a(j,i)=dfroms(lne(idx1(j):idx2(j)),1)
          enddo
        endif
      enddo
      call gpnln2(iu,il,lne,MXLNE,idx1,idx2,nf,lbuf,'rdsqmt')
      if((nf.ne.1).or.(lne(idx1(1):idx2(1)).ne.'end')) then
        call fatal('read_sqmat: <end> expected')
      endif
        ....
      end
\end{verbatim}
Here {\tt iu } is the current input unit , {\tt lne} is a buffer holding 
the input line being processed, {\t nf}  is the number of fields in the line
after splitting it on whitespace, {\tt idx1} points to the beginnings of these 
fields, {\tt idx2} to their ends. Comments and blank lines are 'eaten' by
{\tt gpnln2}, so subsequent call result in new 'data-lines'. 
Mentioning the name of the keyword in the call to {\tt gpnln2} is just to
include it in error messages possibly arising while the line is processed. 

\subsection{Using 'file' instead of 'open'}
The interpreter uses calls to {\tt file} to open input files. This routine 
{\em assigns} FORTRAN unit numbers and keeps track of them. It is possible 
to limit the unit numbers {\tt file} assigns by changing {\tt MINUNT} and {\tt MXUNT} 
in {\tt files.inc}. As an  alternative, calls to {\tt open} can be replaced by calls
to {\tt file}. 


\section{Miscellaneous}
\subsection{Changing the interpreter's behaviour}
The following directives are understood: (all beginning with a dot!)
Default settings marked with a star.

\begin{tabular}{ll}
 directive & meaning \\
 .VERSION & print version of the interpreter \\ 
 .ABBREV  & keywords can be abbreviated * \\
 .NOABBREV & keywords must be spelled out \\
 .SEP      & print the keyword and a separator before calling *\\
 .NOSEP    & dont \\
 .NODEFAULTS & dont allow omitting keyword arguments with defaults\\
 .DEFAULTS & allow omitting keyword arguments * \\
 .VARDEF\_SILENT & use default values in macro retrievals without notice\\
 .VARDEF\_WARN & issue warnings if undefined macros are retrieved *\\
 .VARDEF\_ERROR & make the above fatal errors \\
 .DEBUG & print (lots of) debugging info \\
 .NODEBUG & dont print debugging info * \\
 .COMMENTSc & change the character which starts one line  comments to some\\
 &              value c (if the hash sign is a bad choice) \\
 .ONLYc & if set, {\em only} lines starting with some character c will be
 parsed\\
 .ALL & reset from .ONLY \\  
\end{tabular}
\subsection{Utility keywords}

\begin{tabular}{ll}
printmacros     & print list of currently defined macros \\
savemacros file & save macros as def-statements to file  \\
skipto x        & ignore input until it contains xxx     \\
loop var a b [s]& loop from a to b (step s if given)     \\
? or ?? & print list of keywords (? short, ?? long format)\\
load  file      & load file (nested input) \\
  @@            & floating point operation\\
  @             & integer operation 
\end{tabular}



\end{document}   
 


